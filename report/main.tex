\documentclass[a4paper,12pt]{article}
\usepackage[utf8]{inputenc}
\usepackage[spanish]{babel}
\usepackage{amsmath}
\usepackage{graphicx}
\usepackage{geometry}
\geometry{margin=1in}
\usepackage{hyperref}
\hypersetup{
    colorlinks=true,
    linkcolor=blue,
    filecolor=magenta,      
    urlcolor=cyan,
    pdftitle={Simulación de Eventos Discretos},
    pdfpagemode=FullScreen,
}
\title{Simulación de Eventos Discretos: Reparaciones Electrónicas}
\author{Abraham Romero Imbert}
\date{\today}

\begin{document}

\maketitle

\section{Introducción}

\subsection{Breve descripción del proyecto}

El presente proyecto tiene como objetivo modelar y analizar el sistema de reparación de 
una empresa de electrónica mediante simulación de eventos discretos. La empresa recibe 
aparatos electrónicos de minoristas de la región, los cuales son inspeccionados y clasificados 
para su reparación o envío al fabricante. Este modelo permitirá evaluar el desempeño del sistema, 
identificando métricas clave como el tiempo promedio de procesamiento, la carga de trabajo en 
cada sección y el tiempo total que un aparato permanece en la empresa. Los resultados obtenidos 
servirán para proponer mejoras en la eficiencia operativa del sistema.

\subsection{Objetivos y metas}



\end{document}